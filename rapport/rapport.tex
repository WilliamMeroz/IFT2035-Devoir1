\documentclass{article}
\usepackage[french]{babel}
\usepackage[utf8]{inputenc}

\title{Rapport TP 1}
\author{Laurence Leblond \\ William Méroz-Moreau}

\begin{document}
	\maketitle

	\section{Problèmes rencontrés}

	\subsection{Connaissance trop minime de Haskell}

	Haskell n'est pas un langage très intuitif à prime abord. Penser et déconstruire
	un problème par récursions semble être un défi de taille pour ceux qui n'y ont
	pas l'habitude. Mais à travers les différents problèmes rencontrés, notamment dans
	ce TP et les devoirs, en élargissant pas à pas notre sphère d'aisance, nous sommes
	arrivés à avoir une bonne prise en main de la logique du langage. Tel que
	mentionné dans le cours, il est possible (et souhaitable) de considérer un code
	comme une preuve mathématique, en établissant systématiquement le chemin des
	déductions qui partent du ou des types d'entrée jusqu'au type de sortie, et Haskell
	s'y prête grandement. En ayant en tête l'idée de preuve mathématique (notamment
	par induction), la pensée par récursion devient plus simple: on débute par le
	cas de base, puis par un cas $n$ et $n+1$ qui représentent les autres cas de figures.
	Penser de telle manière est pratique pour bien comprendre le fonctionnement
	des fonctions et pour anticiper les erreurs. Nous avons donc essayé de suivre cette
	démarche logique et systématique dans notre TP, bien que ce ne soit pas un
	muscle à l'usage familier. En gardant ce train de penser, nous avons pu garder
	en tête tous les scénarios possibles qui devaient être couverts, afin d'éviter
	de rendre notre code, et de ce fait notre évaluateur, plus robuste.

	\subsection{Compréhension de la donnée}

	Pour bien comprendre l'énoncé du TP, nous avons décidé de le lire chacun de notre
	côté puis de se rencontrer pour résumer les différentes parties. En
	déconstruisant le travail de la sorte, nous avons mieux compris les différentes
	tâches que nous avions à faire. Il nous a d'abord été compliqué de comprendre
	les exemples de code Slip donnés dans l'énoncé, la syntaxe Lisp étant nouvelle
	pour nous. Il nous a également pris du temps à comprendre ce qui était demandé
	de nous.

	\subsection{Lire, trouver et comprendre les parties importantes du code fourni}

	Pour ce qui est du code donné, la longueur du fichier nous a d'abord parut
	inquiétante et nous avons surévalué la tâche à faire. Mais en ayant bien défini
	les sections à compléter grâce à notre synthèse de la donnée, nous avons
	simplement considéré de gros bouts du code comme des boites noires pour nous
	concentrer exclusivement sur $s2l$ et $eval$. [développer] Avoir travailler sur
	la construction d'évaluateurs dans les différents devoirs nous a permis de comprendre
	ce que nous devions faire dans $eval$ assez rapidement, mais nous avions
	quelques doutes sur la manière dont fonctionnait $s2l$ et comment était défini
	la structure des $sexp$ et des $lexp$. Nous avons débuté par lister les cas que
	nous devions effectuer. Pour $s2l$, il était trivial que nous avions à transformer
	les expressions $Snil$, $Ss ym$, $Snum$ et $Snode$ en $Lnum$, $Lbool$, $Lvar$,
	$Ltest$, $Lfob$, $Lsend$, $Llet$ et $Lfix$. Puis pour $eval$, il fallait prendre
	ces derniers cas et les transformer en $Vnum$, $Vbool$, $Vbuiltin$ et $Vfob$.
	Avant de commencer à coder, nous voulions comprendre la logique des transformations
	(et particulièrement $s2l$). Les cas les plus simples étaient ceux $Snum$ et
	des $Ssym$ que nous comprenions assez bien, mais pour les autres, il nous
	semblait très difficile de trouver un point de départ. Que doit faire $s2l$ au
	juste? Comment sera t-il possible d'interpréter un $sexp$ en $lexp$ pour des
	booléens, des fonctions, etc.? Nous pensions déjà en termes sémantiques, sur la
	valeur des arguments, des opérations, etc. Nous faisions fausse route et cela
	nous a conduit à considérer le travail comme beaucoup plus difficile qu'il ne
	l'était. Lorsque nous avions bien intériorisé le fait que $s2l$ manipule tout simplement
	des arbres syntaxique, le problème nous est apparu plus simple. Nous comprenions
	qu'il n'était que question de quelle disposition de symboles $Sexp$ donne telle
	disposition de symboles $Lexp$.

	\subsection{Compléter le code fourni}

	Ainsi le problème bien opérationnalisé, nous avons simplement commencer par les
	cas de figure qui nous semblaient simples, à savoir, les booléens, qui sont
	syntaxiquement simples (avec deux choix: "$true$" ou "$false$") puis ceux qui débutaient
	par un symbole particulier ("$if$", "$fob$", "$let$", "$fix$"). $if$ nous a semblé
	relativement trivial; nous avons trois expressions, une pour l'énoncé conditionnel,
	une pour la condition "true" et une pour la condition "false" et nous
	retournons un ltest avec chacune des expressions transformées en $Lexp$. Pour $l
	et$, la transformation nous semblait aussi assez simple; on prend une assignation
	et un corps d'expression, puis nous retournons tout simplement un $Llet$ avec
	la variable assignée, la transformation de la valeur assignée et la transformation
	du corps de l'expression. La logique nous semblait similaire pour $Lfix$ mais
	de manière récursive sur un ensemble indéterminé d'assignations. Notre approche
	était de d'établir la logique des fonctions $s2l$ puis celle de $eval$ avant de
	tester. Notre méthode nous semblait bonne, mais nous nous sommes rendu compte en
	cours de route de son inéficacité. La plus grande difficulté, par contre, fut
	l'implémentation des paternes nécessaires à l'interprétation des
	fonctionnalités $fob$, $fix$ et les appels de fonctions ($Lsend$). Il nous a
	fallu du temps avant de comprendre la syntaxe même de ces fonctionnalité en
	code Slip. Pour les $fob$, nous n'avions pas compris à quel point il pouvait
	être difficile d'ajouter les bonnes valeurs à l'environnement de la fonction
	en plus de pouvoir retrouver ces valeurs de l'environnement quand vient de le
	temps d'exécuter la fonction. Pour les $Lsend$, le plus dur a été de
	comprendre comme extraire les arguments et les liés aux bonne variables (paramètres)
	de la fonction durant l'appel d'une fonction définie par l'utilisateur. Les appels
	de fonctoins prédéfinies étaient beaucoup plus simples. Le développement des
	fonctionnalités reliées au mot clé $fix$ ont été le plus compliquées. Il fallait
	tout dabord comprendre le contexte dans lequel fix est utilisé, et pour quelle
	raison. Ensuite, nous avons dû essayer de comprendre tous les cas possibles et
	définir des règles où fix est sensé retourner une erreur. Seulement ensuite nous
	avons pu l'implémenter. Le code des fonctions $s2l$ et $eval$ pour $Lfix$ n'est
	pas très intuitif, et il nous a prit un grand nombre d'heures avant de comprendre
	que $fix$ devrait faire l'utilisation de son propre environnement, en plus de
	devoir faire la distinction entre ce qui est une assignation d'une nouvelle
	variable et ce qui est une déclaration d'une nouvelle fonction. $Lfix$ est vraiment
	la fonctionnalité où tout le reste du code est mit en oeuvre pour complété la tâche
	demandée.
	\section{Choix}

	\subsection{Garder les implémentations de fonctions le plus court possible}

	Nous avons tenté de garder les implémentations des différents paternes des
	fonctions $s2l$ et $eval$ le plus court possible afin de ne pas surcharger l'évaluateur
	avec trop de logique qui pourrait au final créer plus de bogues, et rendre ceux-ci
	plus difficile à identifier. Le code Slip ayant une syntaxe et des
	fonctionnalités assez simple, il est possible de déléguer beaucoup des tâches de
	l'analyse syntaxique et de l'évaluation aux mêmes fonctions pour beaucoup de scénarios
	différents. Par exemple, le code des fonctions $s2l$ et $eval$ pour les $Lfob/V
	fob$ et $Llet$ peuvent se résumer à quelques lignes. Par contre, cette façon de
	travailler nous à causer du fil à retorde lors qu’est venu le temps d'implémenter
	la fonctionnalité des $Lfix$. En effet, puisque nos autres fonctions étaient autant
	dépourvues de fonctionnalité, il nous a fallu écrire beaucoup de code afin d'extraire
	les assignations et les noms d'arguments afin de les ajouter dans leur
	environnement. Ces traitements auraient peut-être pu être ajoutés dans les fonctions
	$s2l$ et $eval$ des $Lfob$ directement afin d'éviter de sur complexifier le code
	et mélanger les rôles des différents paternes.

	\subsection{Ne pas respecter toutes les recommandations faites dans l'énoncé}
	Nous avons commencé le TP en essayant de travailler le plus possible en binôme
	afin de s'entre-aider à bien comprendre la logique que nous devions suivre pour
	compléter le code. Au début du travail, ces heures passées à travailler ensemble
	étaient utiles, mais, nous nous sommes rendus compte qu'il nous était plus facile
	de se séparer l'implémentation en terme de fonctionnalité afin de se laisser à
	chacun la possibilité de se plonger dans le code et réfléchir sur son propre temps.
	Nous avons réaliser par contre que beaucoup de fonctionnalité dépendent des
	autres (on peut seulement bien tester les $Lfob$ et $Vfob$ que lorsque la fonctionnalité
	du $Lsend$ fonctionne bien, par exemple). Une autre recomendation était de
	commencer par implémenter l'évaluation du sucre syntaxique pour la création d'une
	nouvelle fonction anonyme. Après une multitude de tentatives d'implémenter cette
	fonctionnalité, nous avons décider de procéder avec l'implémentation de tout
	le code et de ensuite de soucier de l'implémentation du sucre syntaxique. Avec
	le recul, nous avons réaliser que ce fut une erreur puisque l'implémentation
	du sucre syntaxique après avoir écrit tout le reste du code est beaucoup plus
	difficile. Plus précisément, il est difficile de faire la distinction entre un
	paterne de $Lsend$ et un paterne relié au sucre syntaxique. Les 2 expressions sont
	très similaires, avec des $Snode$ imbriqués de façons similaires, et il aurait
	fallu modifier plusieurs parties du code pour le faire fonctionner correctement.
	Malgré tout nos efforts, nous ne sommes pas parvenus à le faire marcher correctement
	sans impacter le reste de la logique et créer des éffets de bord non désirés.
	Si nous avions suivi la recomendation de l'énoncé en commençant pas le sucre
	syntaxique. Nous n'aurions probablement pas eu ce problème.
	\section{Options rejetées}
	\subsection{Inclure les appels de fonctions anonyme dans $Lfob$}
	Durant l'implémentation de la fonctionnalité d'appel d'une fonction anonyme ($L
	fob$), nous avons commencé par essayer d'implémenter cette fonctionnalité directement
	dans le code pour l'interprétation des $Lfob$. Nous n'avions pas compris que
	puisque la syntaxe Slip $((fob (x)(+ x 1)2)$, par exemple, constituait en
	réalité une déclaration et un appel de fonction en même temps, il était plus sage
	d'implémenté cette fonctionnalité dans le code pour les $Lsend$. Après
	plusieurs jous à essayer de l'implémenter dans les $Lfob$, nous avons rejeté cette
	possibilité.
	\section{Surprises}
	\subsection{Gestion des environnements}
	Nous ne nous attentions pas à ce que la gestion des environnements soit autant
	complexe. Slip étant un langage avec une portée statique, nous nous attentions
	à pouvoir simplement ajouter les données nécéssaires à l'environnement de base
	$env0$ pour ensuite avoir accès à toute variable et sa valeur peut importe l'endroit
	de l'appel de nos fonctions. Non seulement cette façon de penser démontre une mauvaise
	compréhension du concept de portée statique, elle est également impossible
	venant du fait que l'envrionnement $env0$ est immutable dans le code Haskell. Nous
	avons dû, après avoir réalisé notre erreur, programmer les $Lfob$ et, plus
	difficilement, les $Lfix$ afin qu'une « sauvegarde » de l'environnement dans lequel
	la fonction a été crée soit faite et soit garder en mémoire dans les $Lfob$. L'appel
	des fonctions avec $Lsend$ s'est fait sans trop de soucis une fois que nous
	avions compris le processus à suivre.

	\subsection{Importance du code avec faible couplage}
	Nous nous sommes vites rendu compte de l'importance d'écrire du code robuste
	qui pouvait être réutilisé pour chaque paternes que nous avons fais pour les
	fonctions $s2l$ et $eval$. En effet, dépendamment du code Slip que nous
	essayons d'interpréter, il est possible que toutes les options de $s2l$ ou $eva
	l$ soient appelées pour accomplir une tâche commune. Beaucoup de fonctions
	vont dépendre des sorties d'autres fonctions. Si le code est bien fait, ces
	opérations devraient se faire sans réel interventions de notre part et tout
	devrait se lier de façon cohésive.

	\section{Conclusion}
	En conclusion, l'implémentation de ce TP nous a forcé a améliorer nos connaissances
	de Haskell et nous a pousser à essayer de comprendre et implémenter des fonctionnalités
	complexes auxquelles nous n'avions jamais été confrontés auparavant. Nous
	avons commis quelques erreurs, comme celles de ne pas respecter les recommandations
	de l'énoncé et possiblement avoir pris trop de temps à comprendre la tâche à faire.
	Nous avons, par contre, développé notre propre méthode de travail qui nous a
	permis de remplir presque toutes les exigences requises, sauf l'ajout du sucre
	syntaxique.
\end{document}