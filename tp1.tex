\documentclass{article}

\usepackage[french]{babel}
\usepackage[utf8]{inputenc}
\usepackage[T1]{fontenc}
\usepackage{amsfonts}            %For \leadsto
\usepackage{amsmath}             %For \text
\usepackage{fancybox}            %For \ovalbox
\usepackage{hyperref}
\usepackage{color}

\DeclareUnicodeCharacter{03BB}{\ensuremath{{\color{black}{\lambda}}}}

\title{Travail pratique \#1}
\author{IFT-2035}

\begin{document}

\maketitle

\newcommand \mML {\ensuremath\mu\textsl{ML}}
\newcommand \kw [1] {\textsf{#1}}
\newcommand \id [1] {\textsl{#1}}
\newcommand \punc [1] {\kw{`#1'}}
\newcommand \str [1] {\texttt{"#1"}}
\newenvironment{outitemize}{
  \begin{itemize}
  \let \origitem \item \def \item {\origitem[]\hspace{-18pt}}
}{
  \end{itemize}
}
\newcommand \Align [2][t] {\begin{array}[#1]{@{}l} #2 \end{array}}

\section{Survol}

Ce TP vise à améliorer la compréhension des langages fonctionnels en
utilisant un langage de programmation fonctionnel (Haskell) et en écrivant
une partie d'un interpréteur d'un langage de programmation fonctionnel (en
l'occurence une sorte de Lisp).  Les étapes de ce travail sont les suivantes:
\begin{enumerate}
\item Parfaire sa connaissance de Haskell.
\item Lire et comprendre cette donnée.  Cela prendra probablement une partie
  importante du temps total.
\item Lire, trouver, et comprendre les parties importantes du code fourni.
\item Compléter le code fourni.
\item Écrire un rapport.  Il doit décrire \textbf{votre} expérience pendant
  les points précédents: problèmes rencontrés, surprises, choix que vous
  avez dû faire, options que vous avez sciemment rejetées, etc...  Le
  rapport ne doit pas excéder 5 pages.
\end{enumerate}

Ce travail est à faire en groupes de 2 étudiants.  Le rapport, au format
\LaTeX\ exclusivement (compilable sur \texttt{ens.iro}) et le code sont
à remettre par remise électronique avant la date indiquée.  Aucun retard ne
sera accepté.  Indiquez clairement votre nom au début de chaque fichier.

Ceux qui veulent faire ce travail seul(e)s doivent d'abord en obtenir
l'autorisation, et l'évaluation de leur travail n'en tiendra pas compte.
Des groupes de 3 ou plus sont \textbf{exclus}.

\newpage
\section{Slip: Une sorte de Lisp}

\begin{figure}
  \begin{displaymath}
    \begin{array}{r@{~}r@{~}l@{~~}l}
      e &::=& n & \text{Un entier signé en décimal} \\
      &\mid& x & \text{Une variable}  \\
      &\mid& (\kw{if}~e~e_{\kw{then}}~e_{\kw{else}}) &
           \text{Expression conditionelle} \\
      &\mid& (e_0~e_1~...~e_n) &
           \text{Un appel de fonction} \\
      &\mid& (\kw{fob}~(x_1~...~x_n)~e) & \text{Une fonction} \\
      &\mid& (\kw{let}~x~e_1~e_2) &
           \text{Déclaration locale simple} \\
      &\mid& (\kw{fix}~(d_1~...~d_n)~e) &
           \text{Déclarations locales récursives} \\
      &\mid& + \mid - \mid * \mid / \mid ... &
           \text{Opérations binaires prédéfinies}
       \medskip \\
      d &::=& (x~e) & %% \mid ((x~x_1~...~x_n)~e)
          \text{Déclaration de variable} \\
      &\mid& ((x~x_1~...~x_n)~e) &
           \text{Déclaration de fonction}
    \end{array}
  \end{displaymath}
  \caption{Syntaxe de Slip}
  \label{fig:syntaxe}
\end{figure}

Vous allez travailler sur l'implantation d'un langage fonctionnel dont la
syntaxe est inspirée du langage Lisp.  La syntaxe de ce langage est décrite
à la Figure~\ref{fig:syntaxe}.
%%
À remarquer que comme toujours avec la syntaxe de style Lisp, les
parenthèses sont significatives({!}).

Les formes \kw{let} et \kw{fix} (nommée ainsi parce qu'elle construit ce
qu'on appelle parfois un ``point fixe'') sont utilisées pour donner des noms
à des définitions locales.  La différence est que \kw{let} est une forme
plus simple qui n'autorise pas les définitions récursives.  Exemple:
\begin{displaymath}
  \Align[c]{(\kw{let}~x~2~(\kw{let}~y~3~(+~x~y)))} \;\;\;\;\leadsto^*\;\;\;\;5
\end{displaymath}
\begin{displaymath}
  \Align[c]{(\kw{fix}~(\Align{(x~2)\\(y~3))} \\\;\; (+~x~y))} \;\;\;\;\leadsto^*\;\;\;\;5
\end{displaymath}
Vu que beaucoup de définitions locales sont des fonctions, la forme \kw{fix}
accepte une syntaxe particulière pour définir des fonctions:
\begin{displaymath}
  \Align[c]{(\kw{fix}~(\Align{(y~10) \\
              (\Align{(\id{div2}~x) \\
                (/~x~2)))}}
            \\\;\;(\id{div2}~y))}
  \;\;\;\;\leadsto^*\;\;\;\;5
\end{displaymath}
Les définitions d'un \kw{fix} peuvent être mutuellement récursives.  Exemple:
\begin{displaymath}
  \Align[c]{
    (\kw{fix}~
    (\Align{
      (\Align{(\id{even}~x)~
        (\kw{if}~(=~x~0)~\id{true}~(\id{odd}~(-~x~1))))} \\
      (\Align{(\id{odd}~x)~
        (\kw{if}~(=~x~0)~\id{false}~(\id{even}~(-~x~1)))))}} \\
    \;\;(\id{odd}~42))}
  \leadsto^*\;\;\;\;\kw{False}
\end{displaymath}

\subsection{Sucre syntaxique}
\label{sec:sucre}

La syntaxe d'une déclaration de fonction est du sucre
syntaxique, et elle est régie par l'équivalence suivante pour les déclarations:
\begin{displaymath}
  \begin{array}{r@{\;\;\;\;\Longleftrightarrow\;\;\;\;}l}
    ((x~x_1~...~x_n)~e) &
        (x~(\kw{fob}~(x_1~...~x_n)~e))
  \end{array}
\end{displaymath}
où \kw{fob} s'appelle ainsi parce qu'il construit un objet fonction
(\emph{function object} en anglais).
Votre première tâche sera d'écrire une fonction \id{s2l} qui va
``éliminer'' le sucre syntaxique, c'est à dire faire l'expansion des formes
de gauche (présumément plus pratiques pour le programmeur) dans leur
équivalent de droite, de manière à réduire le nombre de cas différents
à gérer dans le reste de l'implantation du langage.  Cette fonction va aussi
transformer le code dans un format plus facile à manipuler par la suite.

\subsection{Sémantique dynamique}

Slip, comme Lisp, est un langage typé dynamiquement, c'est à dire que ses
variables peuvent contenir des valeurs de n'importe quel type.  Il n'y
a donc pas de sémantique statique (règles de typage).

Les \emph{valeurs} manipulées à l'exécution par notre langage sont les entiers, les
fonctions, et les listes (dénotées $[]$, et $[v_1~.~v_2]$).  De plus la
notation de listes est étendue de sorte que $[v_1~.~[v_2~.~[v_3~.~v_4]]]$ se
note $[v_1~v_2~v_3~.~v_4]$ et qu'un ``$. []$'' final peut s'éliminer.
Donc $[v_1~v_2]$ est équivalent à $[v_1~.~[v_2~.~[]]]$.

Les règles d'évaluation fondamentales sont les suivantes:
\begin{displaymath}
  \begin{array}{r@{\;\;\;\leadsto\;\;\;}l}
    ((\kw{fob}~(x_1~...~x_n)~e)~v_1~...~v_n) & e[v_1,...,v_n/x_1,...,x_n] \\
    (\kw{let}~x~v~e) & e[v/x] \\
    (\kw{fix}~((x_1~v_1)~...~(x_n~v_n))~e) & e[v_1,...,v_n/x_1,...,x_n]
  \end{array}
\end{displaymath}
La notation $e[v/x]$ représente l'expression $e$ dans un environnement où la
variable $x$ prend la valeur $v$.  L'usage de $v$ dans les règles ci-dessus
indique qu'il s'agit bien d'une valeur plutôt que d'une expression non
encore évaluée.  Par exemple les $v$ dans la première règle indiquent que
lors d'un appel de fonction, les arguments doivent être évalués avant
d'entrer dans le corps de la fonction, i.e. on veut l'appel par valeur.

En plus des règles $\beta$ ci-dessus, les différentes primitives se
comportent comme suit:
\begin{displaymath}
  \begin{array}{r@{\;\;\;\leadsto\;\;\;}l}
    (+~n_1~n_2) & n_1 + n_2 \\
    (-~n_1~n_2) & n_1 - n_2 \\
    (*~n_1~n_2) & n_1 \times n_2 \\
    (/~n_1~n_2) & n_1 \div n_2 \\
    (\kw{if}~\id{true}~e_1~e_2) & e_1 \\
    (\kw{if}~\id{false}~e_1~e_2) & e_2 \\
  \end{array}
\end{displaymath}

Donc il s'agit d'une variante du $\lambda$-calcul sans grande surprise.
La portée est lexicale et l'ordre d'évaluation est présumé être par
valeur, mais vu que le langage est pur, la différence n'est pas très
importante pour ce travail.

\section{Implantation}

L'implantation du langage fonctionne en plusieurs phases:
\begin{enumerate}
\item Une première phase d'analyse lexicale et syntaxique transforme le code
  source en une représentation décrite ci-dessous, appelée \id{Sexp} dans le
  code.  Ce n'est pas encore un arbre de syntaxe abstraite (cela
  s'apparente en fait à XML ou JSON).
\item Une deuxième phase, appelée \id{s2l}, termine l'analyse syntaxique et
  commence la compilation, en transformant cet arbre en un vrai arbre de
  syntaxe abstraite dans la représentation appelée \id{Lexp} dans le code.
  Comme mentionné, cette phase commence déjà la compilation vu que le
  langage \id{Lexp} n'est pas identique à notre langage source.  En plus de
  terminer l'analyse syntaxique, cette phase élimine le sucre syntaxique
  (i.e.~les règles de la forme $...\Longleftrightarrow...$), et doit faire
  quelques ajustements supplémentaire.
\item Finalement, une fonction \id{eval} procède à l'évaluation de
  l'expression par interprétation.
\end{enumerate}
%%
Une partie de l'implantation est déjà fournie: la première ainsi que divers
morceaux des autres.  Votre travail consistera à compléter les trous.

\subsection{Analyse lexicale et syntaxique}

L'analyse lexicale et syntaxique est déjà implantée pour vous.  Elle est
plus permissive %% générale
que nécessaire et accepte n'importe quelle expression de la
forme suivante:
%%
\begin{displaymath}
  \begin{array}{r@{~}r@{~}l}
    e & ::= & n ~|~ x ~|~
     \punc{(}~\{\,e\,\}%% ~[\,.~e\,]
     ~\punc{)}
  \end{array}
\end{displaymath}
%%
\begin{outitemize}
\item $n$ est un entier signé en décimal.  Il est représenté dans l'arbre en
  Haskell par: $\kw{Snum}~n$.
\item $x$ est un symbole qui peut être composé d'un nombre quelconque de
  caractères alphanumériques et/ou de ponctuation.  Par exemple $\punc+$ est
  un symbole, $\punc{<=}$ est un symbole, $\punc{voiture}$ est un symbole,
  et $\punc{a+b}$ est aussi un symbole.  Dans l'arbre en Haskell, un symbole
  est représenté par: $\kw{Ssym}~x$.
\item $\punc{(}~\{\,e\,\}%% ~[\,.~e\,]
  ~\punc{)}$ est une liste d'expressions.  Dans l'arbre en Haskell, les
  listes d'expressions sont représentées par des listes simplement chaînées
  constituées de paires $\kw{Snode}~\id{left}~\id{right}$ et du marqueur de
  fin $\kw{Snil}$.  \id{left} est le premier élément de la liste et
  \id{right} est le reste de la liste (i.e.~ce qui le suit).
\end{outitemize}
%%
Par exemple l'analyseur syntaxique transforme l'expression \texttt{(+ 2 3)}
dans l'arbre suivant en Haskell:
\begin{displaymath}
  \kw{Snode}~(\kw{Ssym}~\str{+})~
  (\kw{Snode}~(\kw{Snum}~2)~
  (\kw{Snode}~(\kw{Snum}~3)~
  \kw{Snil}))
\end{displaymath}
%%
L'analyseur lexical considère qu'un caractère $\punc{;}$ commence un
commentaire, qui se termine à la fin de la ligne.

\subsection{La représentation intermédiaire \id{Lexp}}

Cette représentation intermédiaire est une sorte d'arbre de syntaxe
abstraite.  Dans cette représentation, \kw{+}, \kw{-}, ... sont simplement
des variables prédéfinies, et le sucre syntaxique n'est plus disponible,
donc les fonctions ne prennent plus qu'un seul argument et la forme \kw{fix}
ne peut définir que des variables.

Elle est définie par le type:

{\small
\begin{verbatim}
data Lexp = Lnum Int             -- Constante entière.
          | Lbool Bool           -- Constante Booléenne.
          | Lvar Var             -- Référence à une variable.
          | Ltest Lexp Lexp Lexp -- Expression conditionelle.
          | Lfob [Var] Lexp      -- Construction de fobjet.
          | Lsend Lexp [Lexp]    -- Appel de fobjet.
          | Llet Var Lexp Lexp   -- Déclaration non-récursive.
          -- Déclaration d'une liste de variables qui peuvent être
          -- mutuellement récursives.
          | Lfix [(Var, Lexp)] Lexp
          deriving (Show, Eq)
\end{verbatim}
}

\subsection{L'environnement d'exécution}

Le code fourni défini aussi l'environnement initial d'exécution, qui
contient les fonctions prédéfinies du langage telles que l'addition, la
soustraction, etc.  Il est défini comme une table qui associe à chaque
identificateur prédéfini la valeur (de type \id{Value}) associée.  La valeur
ne sera utilisée que lors de l'évaluation, mais la liste peut aussi être
utile avant, pour détecter l'usage d'une variable non-définie.

\subsection{Évaluation}

L'évaluateur utilise l'environnement initial pour réduire une expression
(de type \id{Lexp}) à une valeur (de type \id{Value}).
Bien sûr, l'environnement initial s'appelle ainsi parce que c'est la valeur
initiale de l'environnement, mais au cours de l'évaluation, cet
environnement sera ajusté en y ajoutant les variables locales et autres
arguments de fonction.

\section{Cadeaux}

Comme mentionné, l'analyseur lexical et l'analyseur syntaxique sont
déjà fournis.  Dans le fichier \texttt{slip.hs}, vous trouverez les
déclarations suivantes:
\begin{outitemize}
\item \id{Sexp} est le type des arbres, il défini les différents noeuds qui
  peuvent y apparaître.
\item \id{readSexp} est la fonction d'analyse syntaxique.
\item \id{showSexp} est un pretty-printer qui imprime une expression sous sa
  forme ``originale''.
\item \id{Lexp} est le type de la représentation intermédiaire du même nom.
\item \id{s2l} est la fonction qui transforme une expression de type
  \id{Sexp} en \id{Lexp}.
\item \id{Value} est le type du résultat de l'évaluation d'une expression.
\item \id{env0} est l'environnement initial.
\item \id{eval} est la fonction d'évaluation qui transforme une expression
  de type \id{Lexp} en une valeur de type \id{Value}.
\item \id{evalSexp} est une fonction qui combine les phases ci-dessus pour
  évaluer une \id{Sexp}.
\item \id{run} est la fonction principale qui lie le tout; elle prend un nom
  de fichier et applique \id{evalSexp} sur toutes les expressions trouvées
  dans ce fichier.
\end{outitemize}

Voilà ci-dessous un exemple de session interactive sur une machine GNU/Linux,
avec le code fourni:
{\small
\begin{verbatim}
% ghci slip.hs
GHCi, version 9.0.2: https://www.haskell.org/ghc/  :? for help
[1 of 1] Compiling Main             ( slip.hs, interpreted )

slip.hs:241:1: warning: [-Wincomplete-patterns]
    Pattern match(es) are non-exhaustive
    In an equation for ‘eval’:
        Patterns not matched:
            [] (Lbool _)
            [] (Lvar _)
            [] (Ltest _ _ _)
            [] (Lfob _ _)
            ...
    |
241 | eval _ (Lnum n) = Vnum n
    | ^^^^^^^^^^^^^^^^^^^^^^^^
Ok, one module loaded.
ghci> run "exemples.slip"
[2,*** Exception: slip.hs:241:1-24: Non-exhaustive patterns in function eval

ghci> 
\end{verbatim}
}

Les avertissements et l'exception levée sont dus au fait que le code
a besoin de vos soins.  Le code que vous soumettez ne devrait pas souffrir
de tels avertissements, et l'appel à \texttt{run} devrait renvoyer la liste
des valeurs décrites en commentaires dans le fichier d'exemples.

\section{À faire}

Vous allez devoir compléter l'implantation de ce langage, c'est à dire
compléter \id{s2l} et \id{eval}.

Vous devez aussi fournir un fichier \texttt{tests.slip}, similaire
à \texttt{exemples.slip}, mais qui contient au moins 5 tests que \emph{vous}
avez écrits (avec en commentaire la valeur de retour que vous pensez devrait
être renvoyée).  Les tests sont évalués sur les critères suivants:
\begin{itemize}
\item Ils doivent bien sûr être corrects.
\item Ils doivent être suffisamment différents les uns des autres (et des
  exemples fournis) pour détecter des erreurs différentes.
\item Votre implantation de Slip doit les exécuter correctement.
\end{itemize}

\subsection{Recommendations}

Je recommande de faire ce travail ``en largeur'' plutôt qu'en profondeur:
compléter les fonctions peu à peu, pendant que vous avancez dans des
exemples de code Slip simples plutôt que d'essayer de compléter tout
\id{s2l} avant de commencer à attaquer la suite.
Ceci dit, libre à vous de choisir l'ordre qui vous plaît.

De même je vous recommande fortement de travailler en binôme
(\url{https://fr.wikipedia.org/wiki/Programmation_en_bin\%C3\%B4me}) plutôt
que de vous diviser le travail, vu que la difficulté est plus dans la
compréhension que dans la quantité de code.

Le code contient des indications des endroits que vous devez modifiez.
Généralement cela signifie qu'il ne devrait pas être nécessaire de faire
d'autres modifications, sauf ajouter des fonctions auxiliaires.  Vous pouvez
aussi modifier le reste du code, si vous le voulez, mais il faudra alors
justifier ces modifications dans votre rapport en expliquant pourquoi cela
vous a semblé nécessaire.

\subsection{Remise}

Pour la remise, vous devez remettre 3 fichiers (\texttt{slip.hs},
\texttt{tests.slip}, et \texttt{rapport.tex}) par la page Moodle (aussi
nommé StudiUM) du cours.  Assurez-vous que le rapport compile correctement
sur \texttt{ens.iro} (auquel vous pouvez vous connecter par SSH).

\section{Détails}

\begin{itemize}
\item La note sera divisée comme suit: 30\% pour \id{s2l}, 25\% pour le
  rapport, 15\% pour les tests et 30\% pour \id{eval}.
\item Tout usage de matériel (code, texte, ...) emprunté à quelqu'un d'autre
  (trouvé sur le web, ...) doit être dûment mentionné, sans quoi cela sera
  considéré comme du plagiat.
\item Le code ne doit en aucun cas dépasser 80 colonnes.
\item Vérifiez la page web du cours, pour d'éventuels errata, et d'autres
  indications supplémentaires.
\item La note est basée d'une part sur des tests automatiques, d'autre part
  sur la lecture du code, ainsi que sur le rapport.  Le critère le plus
  important est que votre code doit se comporter de manière correcte.
  Ensuite, vient la qualité du code: plus c'est simple, mieux c'est.
  S'il y a beaucoup de commentaires, c'est généralement un symptôme que le
  code n'est pas clair; mais bien sûr, sans commentaires le code (même
  simple) est souvent incompréhensible.  L'efficacité de votre code est sans
  importance, sauf si votre code utilise un algorithme vraiment
  particulièrement ridiculement inefficace.
\end{itemize}

\end{document}
